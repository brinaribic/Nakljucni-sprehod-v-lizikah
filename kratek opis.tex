\documentclass[12pt,a4paper]{amsart}
\usepackage[slovene]{babel}
\usepackage[utf8]{inputenc}
\usepackage{amsmath,amssymb,amsfonts}
\usepackage{url}
\usepackage[dvipsnames,usenames]{color}

\newcommand{\program}{FINANČNA MATEMATIKA}
\newcommand{\ime}{Brina Ribič, David Rozman}
\newcommand{\naslovdela}{Naključni sprehodi v grafih oblike lizike}
\newcommand{\letnica}{2022}

\begin{document}
    
\begin{center}
    \thispagestyle{empty}
\noindent{\large
UNIVERZA V LJUBLJANI\\[1mm]
FAKULTETA ZA MATEMATIKO IN FIZIKO\\[5mm]
\program\ -- 1.~stopnja}
\vfill
\end{center}


\begin{center}
{\bf \naslovdela}\\[10mm]
Kratek opis projekta\\[1cm]
\end{center}
\vfill

\begin{center}
    Avtorja: {\large
\ime}\\[2mm]
   \noindent{\large
Ljubljana, \letnica} 
\end{center}
\pagebreak

\section{Opis problema}

V projektni nalogi si bova natančneje ogledala naključne sprehode v grafih oblike lizike.
Za vhodni podatek bova vzela $(m,n)$ graf lizike, kjer je $m$ število vozlišč pri katerih ima graf obliko polnega grafa, 
$n$ pa je število vozlišč, na katerih ima graf obliko poti. Oba dela grafa sta povezana z mostom.
Zanimal naju bo pričakovani čas, da obiščemo vsa vozlišča in čas, da pridemo nazaj v začetno vozlišče.


\section{Matematično ozadje}

\section{Načrt dela}

\end{document}